\documentclass{beamer}
\usepackage[utf8]{inputenc}

\usetheme{Madrid}
\usecolortheme{default}

\usepackage{bibentry}

%------------------------------------------------------------
%This block of code defines the information to appear in the
%Title page
\title[HITL ML/DA Workflows] %optional
{Human-in-the-loop for Machine Learning/Data Analytics Workflows}
\subtitle{Second Presentation}

\author[An Dan Nguyen] % (optional)
{An Dan Nguyen\\dan.nguyen@aalto.fi}

\date[LSCA-2021] % (optional)
{Large-scale Computing and Data Analysis\\14 May 2021}

%End of title page configuration block
%------------------------------------------------------------


%------------------------------------------------------------
%The next block of commands puts the table of contents at the 
%beginning of each section and highlights the current section:

\AtBeginSection[]
{
  \begin{frame}
    \frametitle{Table of Contents}
    \tableofcontents[currentsection]
  \end{frame}
}
%------------------------------------------------------------


\begin{document}
%The next statement creates the title page.
\frame{\titlepage}

%---------------------------------------------------------
%This block of code is for the table of contents after
%the title page
\begin{frame}
\frametitle{Table of Contents}
\tableofcontents
\end{frame}
%---------------------------------------------------------


\section{Human In the Loop Machine Learning}
%---------------------------------------------------------
\begin{frame}
\frametitle{Human In the Loop Machine Learning}

\begin{itemize}
    \item <1-> Human can involve in every steps of machine learning pipeline.
    \item <2-> Data Extraction, Data Integration, Data Cleaning, Data
Annotation and Iterative labeling, Model training and inference.
    \item <3-> Although there are some common theme and techniques, the human in the loop solutions are tailored according to specific situations. 
    \item <4-> $\rightarrow$ Need to understand the details of the problem needed to solve.
\end{itemize}

\end{frame}
%---------------------------------------------------------


\section{Human-Centered Design of Decision-Support Systems}
%---------------------------------------------------------
\begin{frame}{Human-Centered Design of Decision-Support Systems \cite{smith2009human}}
\begin{enumerate}
    \item <1-> Iterative Human-Centered Design Processes.
    \item <2-> Need Analysis $\rightarrow$ User, Context, Task identification and understanding $\rightarrow$ Identify problem $\rightarrow$ Propose solution $\rightarrow$ Repeat.
\end{enumerate}
\end{frame}
%---------------------------------------------------------


\section{Next step}
%---------------------------------------------------------
\begin{frame}{Next step}
\begin{itemize}
    \item Perform the Need Analysis $\rightarrow$ Interviewing Maarit's team to understand the problem or need.
    \item Identify and understand User, Context, Task.
    \item Then identify the problem and potential solution will come later.
\end{itemize}
\end{frame}
%---------------------------------------------------------


%---------------------------------------------------------
\begin{frame}[t, allowframebreaks]
\frametitle{References}
\bibliographystyle{IEEEtran}
\bibliography{ref}
\end{frame}

%---------------------------------------------------------
%Example of the \pause command
% \begin{frame}
% In this slide \pause

% the text will be partially visible \pause

% And finally everything will be there
% \end{frame}
%---------------------------------------------------------


%---------------------------------------------------------
%Highlighting text
% \begin{frame}
% \frametitle{Sample frame title}

% In this slide, some important text will be
% \alert{highlighted} because it's important.
% Please, don't abuse it.

% \begin{block}{Remark}
% Sample text
% \end{block}

% \begin{alertblock}{Important theorem}
% Sample text in red box
% \end{alertblock}

% \begin{examples}
% Sample text in green box. The title of the block is ``Examples".
% \end{examples}
% \end{frame}
%---------------------------------------------------------


%---------------------------------------------------------
%Two columns
% \begin{frame}
% \frametitle{Two-column slide}

% \begin{columns}

% \column{0.5\textwidth}
% This is a text in first column.
% $$E=mc^2$$
% \begin{itemize}
% \item First item
% \item Second item
% \end{itemize}

% \column{0.5\textwidth}
% This text will be in the second column
% and on a second tought this is a nice looking
% layout in some cases.
% \end{columns}
% \end{frame}
%---------------------------------------------------------


\end{document}