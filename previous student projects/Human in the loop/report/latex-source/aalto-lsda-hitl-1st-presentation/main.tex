\documentclass{beamer}
\usepackage[utf8]{inputenc}

\usetheme{Madrid}
\usecolortheme{default}

\usepackage{bibentry}

%------------------------------------------------------------
%This block of code defines the information to appear in the
%Title page
\title[HITL ML/DA Workflows] %optional
{Human-in-the-loop for Machine Learning/Data Analytics Workflows}
\subtitle{First Presentation}

\author[An Dan Nguyen] % (optional)
{An Dan Nguyen \\ dan.nguyen@aalto.fi}

\date[LSCA-2021] % (optional)
{Large-scale Computing and Data Analysis\\April 2021}

%End of title page configuration block
%------------------------------------------------------------


%------------------------------------------------------------
%The next block of commands puts the table of contents at the 
%beginning of each section and highlights the current section:

\AtBeginSection[]
{
  \begin{frame}
    \frametitle{Table of Contents}
    \tableofcontents[currentsection]
  \end{frame}
}
%------------------------------------------------------------


\begin{document}
%The next statement creates the title page.
\frame{\titlepage}

%---------------------------------------------------------
%This block of code is for the table of contents after
%the title page
\begin{frame}
\frametitle{Table of Contents}
\tableofcontents
\end{frame}
%---------------------------------------------------------


\section{Motivation}
%---------------------------------------------------------
\begin{frame}
\frametitle{Motivation}

\begin{itemize}
    \item <1-> Want to investigate human dynamics topics related to Machine Learning/Data Analysis (ML/DA) workflows.
    \item <2-> Bigger theme of interest is the collaboration between human and machine.
    \item <3-> Got the interest when briefly participate in a research project about human-in-the-loop in new Air Traffic Management systems.
    \item <4-> Some of the concepts in these systems can be applied in the ML/DA workflows (e.g. trust, level of automation analysis framework \cite{lundberg2019human}).
\end{itemize}

\end{frame}
%---------------------------------------------------------


\section{Preliminary Plan}
%---------------------------------------------------------
\begin{frame}{Preliminary Plan}
\begin{enumerate}
    \item <1-> A literature review on current status of human-in-the-loop in ML pipelines and workflow-based data analysis. Can potentially reference other fields if there are not much information \cite{lundberg2019human}.
    \item <2-> Based on the literature review, identify one area to investigate further.
    \item <3-> Identify potential improvement(s) in that area.
    \item <4-> Suggest and develop the concepts for the improvements (a proof of concept as a software may not fit in the scope of this course).
    \item <5-> Write and present results.
\end{enumerate}
\end{frame}
%---------------------------------------------------------


\section{Literature Review}
%---------------------------------------------------------
\begin{frame}{Literature Review}
\begin{itemize}
    \item AutoML Systems \cite{shang2019democratizing} \cite{elshawi2019automated} \cite{patel2020smart}
    \item Human-in-the-loop (HITL) pipeline design, framework, survey, human dynamics
        \begin{itemize}
            \item ML HITL Pipelines/Frameworks \cite{van2021towards} \cite{chai2020humanoutlier} \cite{xin2018accelerating} \cite{xin2018helix}
            \item Data HITL Pipelines/Frameworks \cite{li2017human}
            \item Survey, book about HITL in ML/DA Pipelines \cite{xanthopoulos2020putting} \cite{budd2021survey} \cite{chai2020human} \cite{Munro2020human}
            \item Human dynamics \cite{honeycutt2020soliciting}
        \end{itemize}
    \item Others (opinion, personal perspective) \cite{zanzotto2019human} \cite{bezrukavnikov2021neophyte}
\end{itemize}
\end{frame}
%---------------------------------------------------------


\section{Notable Literature}
%---------------------------------------------------------
\begin{frame}{Notable Literature}
\begin{itemize}
    \item <1-> \textit{A survey on active learning and human-in-the-loop deep learning for medical image analysis,}, S. Budd, E. C. Robinson, and B. Kainz
    \item <2-> \textit{Human-in-the-loop techniques in machine learning,} C. Chai and G. Li
    \item <3-> \textit{Human-in-the-loop machine learning,} R. Munro
    \item <4-> \textit{Human-in-the-loop AI:  Requirements on future (unified) air traffic management systems}, J. Lundberg et al.
\end{itemize}
\end{frame}
%---------------------------------------------------------


\section{Preliminary Goals}
%---------------------------------------------------------
\begin{frame}{Preliminary Goals}
\begin{itemize}
    \item <1-> Understand the current state-of-the art in HITL Machine Learning and write a short report and some potential improvements.
    \item <2-> Or compare some systems of HITL Machine Learning (pros and cons).
    \item <3-> Or something different.
\end{itemize}
\end{frame}
%---------------------------------------------------------


\section{Some Questions/Concerns}
%---------------------------------------------------------
\begin{frame}{Some Questions/Concerns}
\begin{itemize}
    \item <1-> The topic seems a little bit theoretical.
    \item <2-> Since time to conduct experiment involving human can be long and may not fit into the scope of the course.
    \item <3-> Same as a proof of concept in software form.
\end{itemize}
\end{frame}
%---------------------------------------------------------


%---------------------------------------------------------
\begin{frame}[t, allowframebreaks]
\frametitle{References}
\bibliographystyle{IEEEtran}
\bibliography{ref}
\end{frame}

%---------------------------------------------------------
%Example of the \pause command
% \begin{frame}
% In this slide \pause

% the text will be partially visible \pause

% And finally everything will be there
% \end{frame}
%---------------------------------------------------------


%---------------------------------------------------------
%Highlighting text
% \begin{frame}
% \frametitle{Sample frame title}

% In this slide, some important text will be
% \alert{highlighted} because it's important.
% Please, don't abuse it.

% \begin{block}{Remark}
% Sample text
% \end{block}

% \begin{alertblock}{Important theorem}
% Sample text in red box
% \end{alertblock}

% \begin{examples}
% Sample text in green box. The title of the block is ``Examples".
% \end{examples}
% \end{frame}
%---------------------------------------------------------


%---------------------------------------------------------
%Two columns
% \begin{frame}
% \frametitle{Two-column slide}

% \begin{columns}

% \column{0.5\textwidth}
% This is a text in first column.
% $$E=mc^2$$
% \begin{itemize}
% \item First item
% \item Second item
% \end{itemize}

% \column{0.5\textwidth}
% This text will be in the second column
% and on a second tought this is a nice looking
% layout in some cases.
% \end{columns}
% \end{frame}
%---------------------------------------------------------


\end{document}